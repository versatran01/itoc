\section{Reducibility}

A \textbf{reduction} is a way of converting one problem to another problem in such a way that a solution to the second problem can be used to solve the first problem.

Reducibility always involves two problems, $A$ and $B$. If $A$ reduces to $B$, we can use a solution to $B$ to solve $A$.

When $A$ is reducible to $B$, solving $A$ cannot be harder than solving $B$ because a solution to $B$ gives a solution to $A$.

{\color{blue} If A is reducible to $B$ and $B$ is decidable, $A$ is also decidable. If $A$ is undecidable and reducible to $B$, $B$ is undecidable.}

\subsection{Undecidable Problems from Language Theory}

Use the undecidability of $A_\TM$ to prove the undecidability of the halting problem by reducing $A_\TM$ to $HALT_\TM$.


\label{lang:HALTTM_UDCDB}
\begin{shaded}
\textbf{ITOC - Theorem 5.1}

\medskip
$HALT_\TM$ is undecidable
\[
HALT_\TM = \{\desc{M, w} \mid \text{$M$ is a TM and $M$ halts on input $w$}\}
\]
\end{shaded}

\begin{mdframed}
\begin{proof}
Proof by contradiction. Assume that $HALT_\TM$ is decidable and show that $A_\TM$ is decidable. The key idea is to show that $A_\TM$ is reducible to $HALT_\TM$.

Assume that a TM $R$ decides $HALT_\TM$. Construct TM $S$ to decide $A_\TM$.

\medskip
$S$ = On input $\desc{M, w}$, an encoding of a TM $M$ and a string $w$:
\begin{enumerate}
\item Run TM $R$ on input $\desc{M, w}$.
\item If $R$ rejects, \textit{reject}.
\item If $R$ accepts, simulate $M$ on $w$ until it halts.
\item If $M$ accepts, \textit{accept}; if $M$ rejects, \textit{reject}.
\end{enumerate}

If $R$ decides $HALT_\TM$, then $S$ decides $A_\TM$. But $A_\TM$ is undecidable, $HALT_\TM$ is also undecidable.
\end{proof}
\end{mdframed}

We will prove a language undecidable by reducing $A$ to that language.

\label{lang:ETM_UDCDB}
\begin{shaded}
\textbf{ITOC - Theorem 5.2}

\medskip
$E_\TM$ is undecidable.
\[
E_\TM = \{ \desc{M} \mid M \text{ is a TM and } L(M) = \emptyset \}
\]
\end{shaded}

\begin{mdframed}
\begin{proof}
Assume that $E_\TM$ is decidable and then show that $A_\TM$ is decidable, a contradiction. We modify $\desc{M}$ to guarantee that $M$ rejects all strings except $w$, but on input $w$ it works as usual. Then use $R$ to determine whether the modified machine recognizes the empty language. The only string the machine can now accept is $w$, so its language will be nonempty iff it accepts $w$.

The modified machine $M_1$ is 

\medskip
$M_1$ = "On input $x$:
\begin{enumerate}
\item If $x \neq w$, \textit{reject}.
\item If $x = w$, run $M$ on input $w$ and \textit{accept} if $M$ does.
\end{enumerate}

Then assume TM $R$ decides $E_\TM$ and construct TM $S$ that decides $A_\TM$ as follows.

\medskip
$S$ = On input $\desc{M, w}$
\begin{enumerate}
\item Use the description of $M$ and $w$ to construct the TM $M_1$.
\item Run $R$ on input $\desc{M_1}$.
\item If $R$ accepts, \textit{reject}; if $R$ rejects, \textit{accept}. 
\end{enumerate}
\end{proof}
\end{mdframed}

\begin{shaded}
\textbf{ITOC - Theorem 5.3}

\label{lang:REGULARTM_UDCDB}
\medskip
$REGULAR_\TM$ is undecidable.
\[
REGULAR_\TM = \{\desc{M} \mid M \text{ is a TM and } L(M) \text{ is a regular language}\}
\]
\end{shaded}

\begin{mdframed}
\begin{proof}
Assume that $REGULAR_\TM$ is decidable by a TM $R$ and use this to construct a TM $S$ that decides $A_\TM$.

The idea is for $S$ to take its input $\desc{M, w}$ and modify $M$ so that the resulting TM recognizes a regular language iff $M$ accepts $w$. Call this machine $M_2$.

{\color{blue} $M_2$ is \textbf{not} constructed for the purposes of actually running it on some input.} 

We construct $M_2$ only for the purpose of feeding its description into the decider for $REUGLAR_\TM$. Once this decider returns its answer, we can use it to obtain the answer to whether $M$ accepts $w$.

Let $R$ be a TM that decides $REGULAR_\TM$ and construct TM $S$ to decide $A_\TM$.

\medskip
$S$ = On input $\desc{M, w}$
\begin{enumerate}
\item Construct TM $M_2$.

$M_2$ = On input $x$:
\begin{enumerate}
\item If $x$ has the form $0^n1^n$, \textit{accept}.
\item If $x$ does not have this from, run $M$ on input $w$ and \textit{accept} if $M$ accepts $w$.
\end{enumerate}
\item Run $R$ on input $\desc{M_2}$
\item If $R$ accepts, \textit{accept}; if $R$ rejects, \textit{reject}.
\end{enumerate}
\end{proof}
\end{mdframed}

Rice's theorem, states that determining \textbf{any property} of the languages recognized by Turing machines is undecidable.

\label{lang:EQTM_UDCDB}
\begin{shaded}
\textbf{ITOC - Theorem 5.4}

\medskip
$EQ_\TM$ is undecidable.
\[
EQ_\TM = \{\desc{M_1, M_2} \mid \text{$M_1$ and $M_2$ are TMs and }  L(M_1) = L(M_2) \}
\]
\end{shaded}

\begin{mdframed}
\begin{proof}
Show that if $EQ_\TM$ were decidable, $E_\TM$ also would be decidable by giving a reduction from $E_\TM$ to $EQ_\TM$.

If one of these languages happens to be $\emptyset$, we end up with the problem of determining whether the language of the other machine is empty.

Let TM $R$ decide $EQ_\TM$ and construct TM $S$ to decide $E_\TM$.


\medskip
$S$ = On input $\desc{M}$:
\begin{enumerate}
\item Run $R$ on input $\desc{M, M_1}$, where $M_1$ is a TM that rejects all inputs.
\item If $R$ accepts, \textit{accept}; if $R$ rejects, \textit{reject}.
\end{enumerate}
\end{proof}
\end{mdframed}

\subsection{Mapping Reducibility}

Being able to reduce $A$ to $B$ by mapping reducibility means that a computable function exists that converts instances of $A$ to $B$. If we have such a conversion function (reduction), we can solve $A$ with a solver for $B$.

\subsubsection{Computable Functions}

\begin{shaded}
\textbf{ITOC - Definition 5.17}

\medskip
A function $f : \Sigma^* \rightarrow \Sigma^*$ is a \textbf{computable function} if some Turing machine $M$, on every input $w$, hatls with just $f(w)$ on its tape.
\end{shaded}

\subsubsection{Formal Definition of Mapping Reducibility}

\begin{shaded}
\textbf{ITOC - Definition 5.20}

\medskip
Language $A$ is \textbf{mapping reducible} to language $B$, written $A \leq_m B$, if there is a computable function $f: \Sigma^* \rightarrow\Sigma^*$, where for every $w$
\[
w \in A \Leftrightarrow f(w) \in B
\]
The function $F$ is called the \textbf{reduction} from $A$ to $B$.
\end{shaded}

To test whether $w\in A$, we use the reduction $f$ to map $w$ to $f(w)$ and test whether $f(w) \in B$.

\begin{shaded}
\textbf{ITOC - Theorem 5.22}

\medskip
If $A\leq_m B$ and $B$ is decidable, then $A$ is decidable.
\end{shaded}

\begin{mdframed}
\begin{proof}
Let $M$ be the decider for $B$ and $f$ be the reduction from $A$ to $B$. We construct a decider $N$ for $A$.

\medskip
$N$ = On input $w$:
\begin{enumerate}
\item Compute $f(w)$
\item Run $M$ on input $f(w)$ and output whatever $M$ outputs.
\end{enumerate}
\end{proof}
\end{mdframed}

\begin{shaded}
\textbf{ITOC - Corollary 5.23}

\medskip
If $A \leq_m B$ and $A$ is undecidable, then $B$ is undecidable.
\end{shaded}

We used a reduction from $A_\TM$ to prove that $HALT_\TM$ is undecidable. This reduction showed how a decider for $HALT_\TM$ could be used to give a decider for $A_\TM$. We can also use a mapping reduction from $A_\TM$ to $HALT_\TM$. We need to present a computable function $f$ that takes input of th form $\desc{M, w}$ and returns output of the form $\desc{M', w'}$, where
\[
\desc{M, w} \in A_\TM \text{ iff }  \desc{M', w'} \in HALT_\TM
\]

The following machine $F$ computes a reduction $f$.

\medskip
$F$ = On input $\desc{M, w}$:
\begin{enumerate}
\item Construct the following machine $M'$.

$M'$ = On input $x$:
\begin{enumerate}
\item Run $M$ on $x$.
\item If $M$ accepts, \textit{accept}.
\item If $M$ rejects, enter a loop.
\end{enumerate}
\end{enumerate}

In general, when we describe a TM that computes a reduction from $A$ to $B$, improperly formed inputs are assumed to map to strings outside of $B$.

No such reduction exists from $A_\TM$ to $E_\TM$.

We can also use mapping reducibility to show that problems are not Turing-recognizable.

\begin{shaded}
\textbf{ITOC - Theorem 5.28}

\medskip
If $A \leq_m B$ and $B$ is Turing-recognizable, then $A$ is Turing-recognizable.
\end{shaded}

\begin{shaded}
\textbf{ITOC - Corollary 5.29}

\medskip
If $A \leq_m B$ and $A$ is not Turing-recognizable, then $B$ is not Turing-recognizable.
\end{shaded}

In a typical application of this corollary, let $A$ be $\overline{A_\TM}$ and it is not Turing-recognizable. Mapping reducibility implies that $A \leq_m B$ means the same as $\overline{A} \leq_m \overline{B}$. To prove that $B$ isn't recognizable, we show taht $A_\TM \leq_m \overline{B}$.


\label{lang:EQTM_NTR}
\label{lang:EQTMC_NTR}
\begin{shaded}
\textbf{ITOC - Theorem 5.30}

\medskip
$EQ_\TM$ is neither Turing-recognizable nor co-Turing-recognizable.
\end{shaded}

\begin{mdframed}
\begin{proof}
First show that $EQ_\TM$ is not Turing-recognizable by showing that $A_\TM$ is reducible to $\overline{EQ_\TM}$. The reducing function $f$ works as follows.

\medskip
$F$ = On input $\desc{M, w}$
\begin{enumerate}
\item Construct the following two machines, $M_1$ and $M_2$.

$M_1$ = On any input:
\begin{enumerate}
\item \textit{Reject}.
\end{enumerate}

$M_2$ = On any input:
\begin{enumerate}
\item Run $M$ on $w$. If it accepts, \textit{accept}.
\end{enumerate}

\item Output $\desc{M_1, M_2}$.
\end{enumerate}

$M_1$ accepts nothing. If $M$ accepts $w$, $M_2$ accepts everything, and so the two machines are note equivalent. Conversely, if $M$ doesn't accept $w$, $M_2$ accepts nothing, and they are equivalent.

\medskip
To show that $\overline{EQ_\TM}$ is not Turing-recognizable, we give a reduction from $A_\TM$ to the complement of $\overline{EQ_\TM}$, which is $EQ_\TM$. Hence we show that $A_\TM \leq_m EQ_\TM$.

\medskip
$G$ = On input $\desc{M, w}$:
\begin{enumerate}
\item Construct the following two machines, $M_1$ and $M_2$.

$M_1$ = On any input:
\begin{enumerate}
\item \textit{Accept}.
\end{enumerate}

$M_2$ = On any input:
\begin{enumerate}
\item Run $M$ on $w$.
\item If it accepts, \textit{accept}.
\end{enumerate}
\item Output $\desc{M_1, M_2}$.
\end{enumerate}
\end{proof}
\end{mdframed}

\begin{shaded}
\textbf{ITOC - Rice's Theorem}

\medskip
Let $P$ be any nontrivial property of the language of a Turing machine. The problem of determining whether a given Turing machine's language has property $P$ is undecidable.

More formally, let $P$ be a language consisting of Turing machine description where $P$ fulfills two conditions. First $P$ is nontrivial - it contains some, but not all, TM descriptions. Second, $P$ is a property of the TM's language - whenever $L(M_1) = L(M_2)$, we have $\desc{M_1} \in P$ iff $\desc{M_2} \in P$. Here, $M_1$ and $M_2$ are any TMs. $P$ is an undecidable language.
\end{shaded}