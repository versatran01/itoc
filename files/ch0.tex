\section{Introduction}


\subsection{Mathematical Notions and Terminology}

\subsubsection{Sequences and Tuples}

A \textbf{sequence} of objects is a list of these objects in some order.

A sequence with $k$ elements is a \textbf{k-tuple}.

\subsection*{Functions and Relations}

The set of possible inputs to a function is called its \textbf{domain}.

The outputs of a function come from a set called its \textbf{range}.

\[
\mapdef{f}{D}{R}
\]

Functions with $k$ arguments is called a \textbf{k-ary function}.

A binary relation $R$ is an equivalence relation if $R$ satisfies three conditions:
\begin{enumerate}
\item $R$ is \textbf{reflexive} if for every $x$, $xRx$;
\item $R$ is \textbf{symmetric} if for every $x$ and $y$, $xRy$ implies $yRx$;
\item $R$ is \textbf{transitive} if for every $x$, $y$, and $z$, $xRy$ and $yRz$ implies $xRz$.
\end{enumerate}

\subsubsection{Graphs}

The number of edges at a particular node is the \textbf{degree} of that node.

No more than one edge is allowed between any two nodes.

For a graph $G$, $(i, j)$ represents an edge between node $i$ and node $j$. Order of $i$ and $j$ doesn't matter in an undirected graph. We can also describe undirected edges using set notation $\{i, j\}$.

A \textbf{simple path} is a path that doesn't repeat any node.

A graph is \textbf{connected} if every two nodes have a path between them. 

A path is a \textbf{cycle} if it starts and ends in the same node.

A \textbf{simple cycle} is one that contains at least three nodes and repeats only the first and last nodes.

A graph is a \textbf{tree} if it is connected and has no simple cycles.

A path in which all the arrows point in the same direction as its steps is called a \textbf{directed path}. 

A directed graph is \textbf{strongly connected} if a directed path connects every two nodes.

\subsubsection{Strings and Languages}

An \textbf{alphabet} ($\Sigma$) is any nonempty \textbf{\textit{finite}} set. 

The members of the alphabet are the \textbf{symbols} of the alphabet.

A \textbf{string over an alphabet} ($s$ or $w$) is a \textbf{\textit{finite}} sequence of symbols from that alphabet.

The string of length zero is called the \textbf{empty string} ($\epsilon$).

A \textbf{language} ($L$) is a set of strings.

$\Sigma^*$ is the set of all strings over $\Sigma$.
{\color{blue} Different from power set, $\Sigma^*$ is usually infinite.}

{\color{blue} $\emptyset$ is a language, $\{\epsilon\}$ is a different language.}

Strings in a language can be infinite, but alphabets have to be finite.

$\Sigma^k$ is the set of strings of length $k$, each of whose symbol is in $\Sigma$. $\Sigma^0 = \{\epsilon\}$.

$\Sigma = \{0, 1\}$ is the alphabet, $\Sigma^1 = \{0, 1\}$ is a set of strings.

A \textbf{problem} is the question of deciding whether a given string is a member of some particular language.

\subsubsection{Boolean Logic}

the \textbf{implication} operation $\rightarrow$ and is 0 if its first operand is 1 and its second operand is 0.
\[
A \rightarrow B \Leftrightarrow \neg A \vee	B
\]

Some examples
\begin{align*}
P \vee Q 			& \; \Leftrightarrow \; \neg(\neg P \wedge \neg Q) \\
P \rightarrow Q 	& \; \Leftrightarrow \; \neg P \vee Q \\
P \leftrightarrow Q & \; \Leftrightarrow \; (P \rightarrow Q) \wedge (Q \rightarrow P) \\
P \oplus Q 			& \; \Leftrightarrow \; \neg(P \leftrightarrow Q)
\end{align*}

